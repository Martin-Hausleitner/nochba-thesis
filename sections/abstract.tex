\begin{spacing}{1}
    \chapter*{Abstract}
\end{spacing}
\begin{wrapfigure}{r}{0.3\textwidth}
    \begin{center}
      \includegraphics[width=0.2\textwidth]{pics/question_mark.png}
    \end{center}
\end{wrapfigure}
Team Nochba has recognized the growing urbanization trend, with 80\% of the world's population expected to live in cities by 2050. This phenomenon will impact cities globally, where many residents, particularly younger generations, do not know their neighbors. Language and cultural differences further contribute to the formation of isolated groups. To address this issue, Team Nochba aims to develop an app that strengthens mutual neighborhood support and fosters active, cohesive communities in city districts, large residential buildings, and apartment blocks.

The app includes a translation function that helps overcome language barriers among community members. This enables users to seek and offer various forms of assistance, such as shopping for elderly neighbors, helping with renovations, fixing a flat tire, searching for a lost pet, or lending tools. The app also serves as a platform to warn neighbors of potential thefts or other concerns. Users can verify their address via location detection or QR codes during registration, with QR codes for communities shared from phone to phone or posted by the city or housing organizations. Users can define the size of their community by setting a radius, within which they can request or offer help through postings. Neighbors can then view and respond to these posts, with the app facilitating communication between help-seekers and helpers.
\lipsum[6]
\newpage
\begin{spacing}{1}
    \chapter*{Zusammenfassung}
\end{spacing}
\begin{wrapfigure}{r}{0.3\textwidth}
    \begin{center}
      \includegraphics[width=0.2\textwidth]{pics/question_mark.png}
    \end{center}
\end{wrapfigure}
Das Team Nochba hat den weltweiten Anstieg der Bevölkerung erkannt, denn bis 2050 werden voraussichtlich 80\% der Weltbevölkerung in Städten leben. Dieses Phänomen wird sich weltweit auf die Städte auswirken. In vielen Städten, insbesondere bei den jüngeren Generationen, verlieren die Menschen den Kontakt zu ihrer Gemeinschaft und kennen ihre Nachbarn nicht. Sprachliche und kulturelle Unterschiede tragen zusätzlich zur Bildung von isolierten Gruppen bei. Um dieses Problem anzugehen, will das Team Nochba eine App entwickeln, die die gegenseitige Unterstützung in der Nachbarschaft stärkt und aktive, engagierte Gemeinschaften in Stadtteilen, großen Wohngebäuden und Wohnblocks fördert.

Die App enthält eine Übersetzungsfunktion, die hilft, Sprachbarrieren zwischen den Mitgliedern der Gemeinschaft zu überwinden. So können die Nutzer unterschiedliche Formen der Nachbarschaftshilfe suchen und anbieten, beispielsweise für ältere Nachbarn einkaufen, bei Renovierungsarbeiten helfen, eine Reifenpanne beheben, ein entlaufenes Haushaustier suchen oder Werkzeuge ausleihen. Die App dient auch als Plattform, um Nachbarn vor möglichen Einbrüchen oder anderen Gefahren zu warnen. Die Nutzer können ihre Adresse während der Registrierung über die Standorterkennung oder QR-Codes verifizieren, wobei QR-Codes für Gemeinschaften von Telefon zu Telefon weitergegeben oder von der Stadt oder Wohnungsbaugenossenschaften veröffentlicht werden. Die Nutzer können die Größe ihrer Community definieren, indem sie einen Radius festlegen, innerhalb dessen sie über Postings Hilfe anfordern oder anbieten können. Die Nachbarn können dann diese Beiträge sehen und darauf antworten, und die App erleichtert die Kommunikation zwischen Hilfesuchenden und Helfern.
\emph{Bitte auf keinen Fall mit der Zusammenfassung verwechseln, die den Abschluss der Arbeit bildet!}
\lipsum[6]
