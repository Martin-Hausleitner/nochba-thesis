\section{Technologieevaluierung}

Im Rahmen dieser Diplomarbeit wurden umfangreiche Anforderungen an das Frontend und Backend der Anwendung gestellt, um eine hochwertige und funktionale Anwendung zu entwickeln. Für das Frontend wurden iOS- und Android-Apps gefordert, wobei die Möglichkeit einer zukünftigen Web-App berücksichtigt werden sollte. Das Backend sollte eine Echtzeit-Datenbank für den Chat und eine Datenbank mit Geofunktionen für den Radios-Filter bereitstellen. Dabei war es wichtig, dass das Backend einfach zu implementieren ist und eine gute SDK-Unterstützung für das Frontend bietet.

Neben diesen technischen Anforderungen war auch die Umsetzung der Business-Logik ein wichtiger Aspekt. Hierbei wurden Aspekte wie die Registrierung und Verifizierung von Nutzern berücksichtigt, um eine sichere und benutzerfreundliche Anwendung zu gewährleisten. Darüber hinaus war es wichtig, dass die Anwendung mit neuen und effizienten Technologien entwickelt wird, um eine schnelle und agile Anpassung an die sich ändernden Anforderungen des Marktes und der Nutzer zu ermöglichen.


\subsection{Frontend}

Nach ausführlicher Recherche wurden die folgenden plattformübergreifenden Frameworks in Betracht gezogen: Flutter, React Native und Xamarin. Jede Plattform hat ihre Vor- und Nachteile.

Flutter, das von Google entwickelt wird, hat eine einfache Syntax und ist leicht zu erlernen. Es bietet auch schnelles Debugging und Hot-Reload-Funktionen, was die Entwicklung schneller und effizienter macht. Ein weiterer Vorteil von Flutter ist die Verfügbarkeit vieler Pakete und Bibliotheken, die die Implementierung vereinfachen können. Flutter bietet auch eine hervorragende Integration mit VS Code, einschließlich Refactoring-Funktionen und der Möglichkeit, Widgets in separate Klassen aufzuteilen.

Laut einer Studie von Raul Hernandez und Pablo Suarez von der Universidad de Murcia in Spanien zeigt Flutter eine höhere Leistung als React Native und Xamarin. Die Studie verglich die Frameworks in Bezug auf Kompilierungszeit, Ausführungszeit, Speicherauslastung und APK-Größe und stellte fest, dass Flutter in allen Bereichen am besten abschnitt. \cite{hernandez_suarez_2021}

Nach langer Recherche und Evaluierung aller Faktoren wurde Flutter als Plattform für das Frontend ausgewählt. Insbesondere die Einfachheit der Syntax und das schnelle Debugging sowie die Integration in VS Code waren entscheidende Faktoren. Jeder aus unserem Team hat einen eigen Flutter Test App programmiert \cite{flutter_test_apps}

Insgesamt waren wir mit unserer Entscheidung, Flutter zu verwenden, sehr zufrieden. Während der Entwicklungszeit stellte sich heraus, dass die Klammerverwendung in Flutter am Anfang etwas ungewohnt war, aber dank der hervorragenden Unterstützung durch die Flutter-Community und der umfangreichen Dokumentation auf Stack Overflow konnten wir alle Herausforderungen bewältigen. Außerdem wurde die Entwicklung durch die ständigen Verbesserungen und Updates von Flutter, wie der Einführung der Impeller-Rendering-Engine, beschleunigt. \cite{flutter_impeller}

Insgesamt war die Wahl von Flutter für das Frontend der App eine gute Entscheidung.

\subsection{Backend}

Für das Backend wurden verschiedene Optionen in Betracht gezogen, einschließlich der Verwendung von Cloud-basierten Plattformen wie Firebase oder selbst gehosteten Lösungen wie Node.js mit MongoDB.

Letztendlich wurde Firebase von Google als Backend-Plattform gewählt, da es eine Echtzeit-Datenbank für den Chat und eine Geodatenbank für den Radios-Filter bereitstellt und eine gute Integration mit Flutter bietet. Firebase bietet auch eine einfache Implementierung von Authentifizierungsfunktionen, einschließlich der Möglichkeit zur Registrierung und Verifizierung von Nutzern.

Firebase bietet außerdem eine automatische Skalierung, die die Leistung der Anwendung bei Bedarf erhöhen kann. Auch die Kosten für die Verwendung von Firebase sind im Vergleich zu anderen Optionen relativ gering.

Es gab jedoch auch einige Herausforderungen bei der Verwendung von Firebase, insbesondere im Hinblick auf die Anpassung an spezifische Anforderungen und die Skalierung bei hohem Datenverkehr. Glücklicherweise bietet Firebase eine umfangreiche Dokumentation und eine aktive Community, die bei der Lösung von Problemen helfen kann.

Insgesamt war Firebase eine gute Wahl als Backend-Plattform für die Anwendung, da es alle Anforderungen erfüllte und eine gute Integration mit Flutter bot.

\section{Fazit}

Die Wahl von Flutter als Frontend-Plattform und Firebase als Backend-Plattform erwies sich als gute Entscheidung für die Anforderungen dieser Diplomarbeit. Die Einfachheit der Syntax, das schnelle Debugging und die umfangreiche Dokumentation und Unterstützung durch die Community machten Flutter zu einer einfachen und effizienten Plattform für die Entwicklung der mobilen Apps. Firebase bot eine gute Lösung für die Echtzeit-Datenbank und die Geodatenbank und eine einfache Integration mit Flutter.

Insgesamt haben die gewählten Technologien dazu beigetragen, dass die Entwicklung der Anwendung schnell und effizient verlief und die Anforderungen der Diplomarbeit erfüllt wurden.

