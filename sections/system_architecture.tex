% \lipsum[4] Citing \cite{InfH} properly.

% Was ist eine \gls{guid}?
% Eine \gls{guid} kollidiert nicht gerne.

% Kabellose Technologien sind in abgelegenen Gebieten wichtig \cite{APCW2006}.



\section{Flutter}

Hier kommt die Beschreibung der Technologie und deren Vorteile/Nachteile, auf Basis von wissenschaftlichen Studien und Erfahrungen aus der Praxis.

Quellen: 

https://flutter.dev/docs/resources/technical-overview, https://pub.dev/packages/flutter
% \subsection{iOS}
% \subsubsection{CI/CD}

% \subsection{Android}
% \subsubsection{CI/CD}
% \subsubsection{Firebase App Distribution}




\section{Firebase}

Hier wird die Vorstellung der Firebase Technologie inklusive deren Funktionen und Vorteile erklärt.

Quellen: 

https://firebase.google.com/docs/guides 

https://medium.com/flutter-community/flutter-firebase-from-scratch-28c8ba7d98b5

\subsection{Firebase Authentication}

Beschreibung der Firebase Authentication Technologie, inklusive Sicherheitsregeln und Best Practices.

Quellen: 

https://firebase.google.com/docs/auth 

https://medium.com/flutter-community/firebase-authentication-in-flutter-752d14209a8a

\subsubsection{Security Rules}

Hier kommt die Erklärung der Sicherheitsregeln in Firebase Authentication und deren Bedeutung für die App Sicherheit.

Quellen: 

https://firebase.google.com/docs/rules 

https://firebase.google.com/docs/auth/admin/custom-claims

\subsection{Cloud Firestore}

Vorstellung der Datenbanktechnologie Firestore, inklusive Datenmodelle.

Quellen: 

https://firebase.google.com/docs/firestore

https://medium.com/flutter-community/firebase-cloud-firestore-in-flutter-26c6e8c6f90c

\subsubsection{Datenmodel}

Hier kommt die Erklärung des Datenmodells in Firebase Firestore und dessen Auswirkungen auf die App-Architektur.

Quellen: 

https://firebase.google.com/docs/firestore/data-model

https://www.raywenderlich.com/6628345-cloud-firestore-for-flutter-getting-started

Weitere wichtige punkte:

\begin{compactitem}
    \item Präsentation des eigenen Datenmodells in der Arbeit.
    \item Diagramme werden verwendet, um das Modell zu präsentieren und Entscheidungen zu erläutern und zu begründen.
    \item Anforderungen der App-Architektur werden dabei berücksichtigt werden.
    \item Performance, Skalierbarkeit und Strukturierung können thematisiert werden.
    \item Zur Veranschaulichung unserer eigenen Datenmodell-Entwicklung kann auf ein Beispiel-Datenmodell-Diagramm auf der offiziellen Firebase-Website verwiesen werden, welches uns als Orientierungshilfe diente.
\end{compactitem}

Beispiel-Datenmodell-Diagramm Quelle: 

https://firebase.google.com/docs/firestore/data-model\#structure\_your\_data

\subsection{Cloud Storage}

Beschreibung der Cloud Storage Technologie in Firebase und deren Einsatz in der App.

Quellen: 

https://firebase.google.com/docs/storage 

https://medium.com/flutter-community/firebase-cloud-storage-in-flutter-flutter-an-firebase-tutorial-c5de7835c6cd

\subsection{Firebase Cloud Functions}

Erklärung der Cloud Functions Technologie in Firebase, inklusive Beispiele für deren Einsatz in der App.

Quellen: 

https://firebase.google.com/docs/functions 

https://medium.com/codeburst/organizing-your-firebase-cloud-functions-67dc17b3b0da

\section{Algolia Search}

Vorstellung der Algolia Search Technologie und deren Integration in die App-Architektur.

Quellen: 

https://www.algolia.com/doc/ 

https://www.algolia.com/doc/guides/sending-and-managing-data/send-and-update-your-data/tutorials/firebase-algolia/

\subsubsection{Firebase Cloud Function}

Beschreibung der Firebase Cloud Functions und deren Rolle in der Algolia Integration.

Quellen: 

https://firebase.google.com/docs/functions

https://www.algolia.com/doc/guides/sending-and-managing-data/send-and-update-your-data/tutorials/firebase-algolia/
