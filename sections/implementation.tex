\section{Technologieevaluierung}

In dieser Arbeit soll eine iOS- und Android-App entwickelt werden, die in Zukunft auch als Web-App verfügbar sein soll. Das Backend soll eine Echtzeit-Datenbank für den Chat sowie eine Datenbank mit Geofeatures für unseren Radios-Filter beinhalten und leicht zu implementieren sein. Des Weiteren soll das Backend die Geschäftslogik für die Registrierung und Verifizierung von Benutzern beinhalten. Gute SDKs für das Frontend sind ebenfalls wichtig.

\subsection{Frontend}

Nach langen Recherchen kamen folgende plattformunabhängige Frameworks in Frage: Flutter, React Native und Xamarin. Es gibt Unterschiede zwischen diesen Frameworks, jedoch ist Flutter aufgrund der folgenden Gründe die beste Wahl für unsere Arbeit:

\begin{itemize}
\item Schönerer Syntax: Flutter hat eine Syntax, die der von CHARO ähnelt, was uns an der HTL beigebracht wurde.
\item Straightforward: Flutter ist einfach zu verstehen und zu erlernen.
\item Hot Reload: Das Debugging ist durch das Hot-Reload-Feature wesentlich einfacher und schneller.
\item Fertige Packages: Flutter verfügt über eine Vielzahl an Paketen, die das Programmieren erleichtern.
\item Schnelle Entwicklung: Durch Flutter können Apps schneller entwickelt werden.
\end{itemize}

Im Nachhinein stellte sich die Entscheidung, Flutter zu verwenden, als richtig heraus. Zwar war die Verwendung von Klammern am Anfang nervig und unübersichtlich, jedoch gibt es für Visual Studio Code eine sehr gute Erweiterung für Flutter \cite{flutter_extension}, die das Programmieren wesentlich einfacher und schneller macht. Die Flutter-Community ist sehr aktiv und es gibt viele Ressourcen, die bei der Lösung von Problemen helfen können.

Wir haben während der Entwicklung jeder einen Prototypen mit Flutter erstellt \cite{flutter_test_repo}, was uns geholfen hat, schnell ein Gefühl für die Plattform zu bekommen. Seit Beginn der Entwicklung im Jahr 2022 hat sich Flutter von Version 2.10 auf Version 3.7 entwickelt. In dieser Zeit wurden viele kleine Änderungen vorgenommen, die die Entwicklung schneller und einfacher gemacht haben. Beispielsweise wurde die "Impeller"-Engine entwickelt, die den Renderprozess beschleunigt \cite{flutter_impeller}.

Insgesamt ist Flutter aufgrund seiner benutzerfreundlichen Syntax, der Hot-Reload-Funktion, der Fülle an Paketen und der schnellen Entwicklung die beste Wahl für unser Frontend.

\bibliographystyle{plain}
\bibliography{references}

\end{document}