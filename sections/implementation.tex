\section{Continuous Integration/Delivery}
\subsection{GitHub Actions}

allg. actions warum kosten
\subsubsection{Build IOS}
\setauthor{Martin Hausleitner}

\subsubsection{Build Android}

\subsection{Fastline}
\subsubsection{Build Number increment}
\subsection{Firebase App Distribution}

\section{Mobile Anwendung}
\subsection{Dateistruktur}
\setauthor{Martin Hausleitner}
In Flutter gibt es keine fixe Dateistruktur für eine App,
man kann seine Struktur also selbst überlegen und gestalten.
Im Folgenden beschreibe ich, wie wir unsere Dateistruktur
für eine Flutter-App aufgebaut haben.

In Flutter gibt es keine feste Dateistruktur, stattdessen kann man die Struktur der Dateien und Ordner selbst bestimmen. Für unser Flutter-Projekt haben wir uns für eine Struktur entschieden, die sich an bewährten Praktiken orientiert.

Unsere Dateistruktur sieht wie folgt aus:

\begin{itemize}
  \item \textbf{logic} - Hier befindet sich die Geschäftslogik der App, einschließlich der Firestore-Cloud-Funktionen und Repositories, die API-Aufrufe ausführen.
  \item \textbf{pages} - Hier werden Widgets entworfen, die jeweils eine Seite der App darstellen.
  \item \textbf{routes} - Hier werden die Routen definiert, die tiefere Links ermöglichen.
  \item \textbf{shared} - Hier werden UI-Widgets wie Buttons oder andere Widgets gespeichert, die oft wiederverwendet werden.
  \item \textbf{views} - Hier befinden sich Ansichten, die von mehreren Seiten der App verwendet werden können.
\end{itemize}

Im Nachhinein hätten wir die Dateistruktur anders gestaltet,
z.B. hätten wir das UI als eigenes Package definiert und die
pages und views besser unterteilt.

\subsection{State Management}
In Flutter gibt es verschiedene Möglichkeiten\cite{flutter-docs-interactive}\cite{flutter-state-management-blog}, um mit dem State Management umzugehen. State Management bezieht sich auf die Art und Weise, wie Daten innerhalb einer App verwaltet werden. In jeder App gibt es bestimmte Daten, die von verschiedenen Komponenten und Widgets verwendet werden und sich im Laufe der Zeit ändern können. State Management bezieht sich auf die Methoden, die verwendet werden, um diese Daten innerhalb der App zu verwalten und zu aktualisieren.
\setauthor{Martin Hausleitner}

\subsubsection{GetX}
\setauthor{Martin Hausleitner}

GetX verwendet ein reaktives Ansatz zur Verwaltung des Zustands, was bedeutet, dass Änderungen im Zustand automatisch die UI aktualisieren, ohne dass der Entwickler manuell Code schreiben muss, um diese Aktualisierungen durchzuführen. Dies spart viel Entwicklungszeit und macht es einfach, auf Benutzerinteraktionen zu reagieren.

Mit GetX können wir auch eine einheitliche Datenquelle haben, auf die alle Komponenten zugreifen können, was die Wartung und Erweiterung der Anwendung erleichtert. Darüber hinaus bietet GetX auch eine einfache Möglichkeit, Abhängigkeiten zu verwalten und Zustandsinformationen zwischen Bildschirmen zu teilen.

Insgesamt hat uns die Verwendung von GetX im Flutter-Framework sehr geholfen, eine effektive und skalierbare Anwendung zu erstellen, die auf die Bedürfnisse unserer Benutzer abgestimmt ist.
\setauthor{Sandin Habibovic}
getx controller service etc...


\subsection{Authentifizierung}
\setauthor{Sandin Habibovic}

\subsubsection{Anmelde Flow}
\setauthor{Sandin Habibovic}

Diagram
erklärung
screenshots
\subsubsection{Regestrierungs Flow}
\setauthor{Sandin Habibovic}


Diagram
erklärung
screenshots


\subsubsection{Firebase Authentifizierung}
\setauthor{Sandin Habibovic}

allg.


\subsection{Feed}
\setauthor{Sandin Habibovic}
Die Feed-Page ist die erste und wichtigste Anzeige auf der App. Hier werden alle aktuellen Beiträge und Updates von der Nachbarschaft angezeigt.

\subsection{Beiträge}
\setauthor{Sandin Habibovic}
Beiträge sind das Hauptkommunikationsmittel auf der App. Jedem Beitrag muss ein Titel, eine Beschreibung und eine Reichweite, unter der, der Beitrag sichtbar ist, angegeben werden. Weiters ist es möglich einem Beitrag ein Bild und Tags anzuhängen.

\subsubsection{Kategorien}
\setauthor{Sandin Habibovic}
Um Beiträge besser zuordnen zu können, muss der User den Beitrag vor dem Veröffentlichen in eine bestimmte Kategorie einteilen. Diese Kategorien ermöglichen es Usern, die Art Ihrer Anfrage ihm vorhinein besser zu spezifizieren und die Suche nach Beiträgen einer bestimmten Art zu vereinfachen. Bestimmte Kategorien werden weiters in Unterkategorien aufgeteilt, da diese ein zu weit gefächertes Genre an Anfragen umfassen.

Es existieren folgende Kategorien bzw. Unterkategorien:

\begin{compactitem}
  \item Mitteilung
  \begin{compactitem}
    \item Frage
    \item Appell
    \item Warnung
    \item Empfehlung
    \item Gefunden
  \end{compactitem}
  \item Suche
  \begin{compactitem}
    \item Hilfe
    \item Verloren
  \end{compactitem}
  \item Ausleihen
  \item Event
\end{compactitem}



Mitteilung:
Die Kategorie der Mitteilung dient dazu die Nachbarn über ein bestimmtes Ereignis oder Meldung zu informieren oder zu befragen.

Suche:
Die Kategorie der Suche dient dazu mit den Nachbarn im Falle einer Hilfesuche oder eines verloren gegangenen Objekts in Kontakt zu treten.

Ausleihen:
Die Kategorie des Ausleihens dient dazu die Nachbarn nach der Erlaubnis, sich ein bestimmtes Werkzeug oder Objekt ausborgen zu dürfen, zu bitten.

Event:
Die Kategorie des Events dient dazu die Nachbarn auf eine bestimmte Veranstaltung aufmerksam zu machen.


\subsubsection{Tags}
\setauthor{Sandin Habibovic}
Als Tag wird ein Schlüsselwort beschrieben, was man an ein Informationsgut anhängen kann, um es besser beschreiben zu können und/oder besser auffindbar zu machen. In der App werden Tags als eine Erweiterung der Kategorien verwendet, um es Usern zu ermöglichen Ihren Beitrag einem selbstdefinierten Typ zuzuordnen.

\subsubsection{Info}
\setauthor{Sandin Habibovic}
Jeder Beitrag hat eine eigene Sektion, wo wichtige Entscheidungsinformationen angegeben werden, wie der Stadtteil und die ungefähre Entfernung zum gegebenen Nachbarn und das Erstelldatum des Beitrags.

\subsubsection{Kommentare}
\setauthor{Sandin Habibovic}
Die Kommentarfunktion ermöglicht es den Usern unter einem Beitrag Ihre Meinung, Feedback oder sonstiges zu hinterlassen.

\subsubsection{Beitrag oder Kommentar Melden}
\setauthor{Sandin Habibovic}
Um auf unangebrachte Beiträge oder Kommentare schnell reagieren zu können, gibt es die Möglichkeit Beiträge oder Kommentare zu melden. Diese Meldungen werden auf Firestore gespeichert und können dann im Einzelnen überprüft werden. Fürs Melden muss ein Grund ausgewählt und eine genauere Beschreibung angegeben werden.
Gründe fürs Melden eines Beitrags oder Kommentars:

\begin{compactitem}
  \item Unangebrachter Inhalt
  \item Belästigung
  \item Betrug
  \item Spam
  \item Sonstiges
\end{compactitem}

\subsection{Filter}
\setauthor{Sandin Habibovic}
Der Filter bietet die Option die Beiträge nach bestimmten Kriterien zu filtern und die Suche nach bestimmten Beiträge zu vereinfachen. Der Filter unterteilt sich in einen Menüfilter und Hauptfilter.
Die Ansicht vom Menüfilter befindet sich direkt über den Beiträgen und ermöglicht eine schnelle Filterung nach einzig allein den Hauptkategorien.
Die Ansicht vom Hauptfilter taucht erst nach dem Antippen vom Filtersymbol auf und beinhaltet eine größere Auswahl an Filteroptionen. Darunter zählt neben dem Filtern nach Hauptkategorien auch die zusätzliche Möglichkeit genauer nach Unterkategorien zu suchen. Außerdem besteht auch die Option die Beiträge nach Datum oder Likes zu sortieren oder die Beiträge aufsteigend oder absteigend zu ordnen. Die wichtigste Filterkomponente ist der Range-Slider, womit die Beiträge nach der Reichweite gefiltert werden können, da die Entfernung zum Nachbarn eine der wichtigsten Entscheidungsfaktoren zum Antworten auf einem Beitrag ist.

\subsection{Suche}
\subsection{Typesense}
\subsection{Algolia}
diagram
\subsubsection{Firestore Sync}

\subsubsection{Algolia SDK}


\subsection{Chat}
package genommen warum
\subsubsection{Flyer Package}

\subsection{Profil}
\setauthor{Sandin Habibovic}
Die Profilanzeige ist die öffentliche Informationsstelle über den User. In dieser Anzeige werden als erster Eindruck der Name und das Profilbild vom User angezeigt. Genauere Informationen über den User können im Bereich Nutzerinfo gefunden werden. Darunter zählen:
\begin{compactitem}
  \item Geburtstag
  \item Beruf
  \item Bio
\end{compactitem}
Außerdem beinhaltet die Anzeige eine eigene Beitragssicht, wo alle Beiträge vom jeweiligen User eingesehen werden können.

\subsubsection{Profil Melden}
\setauthor{Sandin Habibovic}
Falls das Profil vom User unangebrachten Inhalt aufweist, besteht die Möglichkeit das Profil zu melden.

\subsection{Benachrichtigungen}
\setauthor{Sandin Habibovic}
Benachrichtigungen dienen dazu die Nachbarn über mögliche Hilfsbereitstellungen zu informieren, bevor der Kontakt überhaupt entsteht. Die Benachrichtigung zeigt den Nachbarn an, der in Kontakt treten möchte, und möglicherweise den Beitrag auf dem geantwortet wurde. Außerdem ist die Benachrichtigung mit einem „Annehmen“- und „Ablehnen“- Button ausgestattet, womit das Hilfsangebot angenommen oder abgelehnt werden kann.

\subsection{Einstellungen}
\setauthor{Sandin Habibovic}
Die Einstellungssicht der Anwendung bietet eine Reihe an nützlichen Funktionen, die dem User mehr Kontrolle über sein Konto geben.
Dazu gehört zu einem die Option, die Sprache der App umzustellen, um dem User zu ermöglichen, die Anwendung in der bevorzugten Sprache zu nutzen. Zum derzeitigen Stand kann die App sich in zwei Sprachen übersetzen lassen: Deutsch und Englisch.
Darüber hinaus können Benutzer auch entscheiden, ob sie Benachrichtigungen erhalten möchten oder nicht, und diese Einstellungen jederzeit ein- oder ausschalten. Diese Einstellungen bieten den Benutzern eine höhere Privatsphäre und Personalisierungsmöglichkeiten, um die Anwendung besser an ihre individuellen Bedürfnisse anzupassen.
Zu den weiteren Einstellungen gehört das Umändern der E-Mail oder des Passworts, um die Sicherheit des Kontos zu gewährleisten und unbefugten Zugriff zu verhindern.
Die letzte Funktion in der Einstellungssicht ist das Löschen des eigenen Kontos, womit alle Daten des Users gelöscht werden. Vor dem endgültigen Löschen des Kontos wird der User allerdings aufgefordert, seine Entscheidung zu bestätigen.

\subsection{Feedback}
\setauthor{Martin Hausleitner}
Im Zuge unserer Testphase suchen wir nach einer effizienten Methode, um Feedback zu sammeln. Um ein fundiertes Verständnis einer Fehlermeldung zu erlangen, erweist sich die Bereitstellung eines Screenshots als besonders nützlich. Daher haben wir uns nach Packages umgesehen, welche die Möglichkeit bieten, Screenshots zu erstellen und ergänzende Informationen als Kontext hinzuzufügen. Wir haben uns für das Package https://pub.dev/packages/feedback entschieden. Dieses Package gestattet dem Nutzer, einen Screenshot anzufertigen und Annotationen hinzuzufügen, bevor ein entsprechender Kontext in Textform eingefügt werden kann. Eine weitere praktische Funktion besteht darin, dass der Nutzer bei Bedarf noch einmal navigieren und den Screenshot wiederholen kann. Zur Initiierung dieses Feedback-Views haben wir das Package https://pub.dev/packages/shake genutzt, welches bei einer entsprechenden Bewegung des Mobilgeräts aktiviert wird. Wir haben uns bei dieser Funktion an anderen Anwendungen orientiert, die ähnliche Mechanismen für die Feedback-Erhebung nutzen.

Das von uns gewählte Vorgehen zur Analyse des Feedbacks erfolgt über die Trello-API, welche es uns ermöglicht, ein entsprechendes Dashboard zur Erfassung und Bearbeitung der Meldungen zu erstellen. Die erfassten Informationen, wie beispielsweise der Screenshot, der Text, das Datum und relevante Systeminformationen, dienen uns als Grundlage für eine weitere Auswertung und Fehlerbehebung. Trello erleichtert uns die Verwaltung des Feedbacks auf kostenlose Weise und stellt somit ein adäquates Instrument zur Unterstützung des Entwicklungsprozesses dar.


\section{UI/UX Design}
Das UI/UX-Design einer App ist von entscheidender Bedeutung für den Erfolg der App und die Zufriedenheit der Benutzer. Insbesondere bei einer Social-Media-Nachbarschafts-App ist ein gut durchdachtes UI/UX-Design unerlässlich, um eine positive Benutzererfahrung zu gewährleisten.

Ein gutes UI-Design ist wichtig, um sicherzustellen, dass die Benutzer die App einfach und intuitiv bedienen können. Es sollte eine klare Struktur und Navigation haben, damit die Benutzer schnell zu den gewünschten Funktionen gelangen können. Wenn die Benutzer die App als kompliziert oder verwirrend empfinden, werden sie möglicherweise frustriert und geben die Nutzung der App auf.

Auch ein gutes UX-Design ist wichtig, um sicherzustellen, dass die Benutzer mit der App zufrieden sind. Es sollte eine ansprechende und ansprechende Benutzeroberfläche bieten, die dem Benutzer ein angenehmes Nutzungserlebnis vermittelt. Wenn die Benutzer die App als langweilig oder uninteressant empfinden, werden sie möglicherweise nicht wiederkommen.

Darüber hinaus sollte das UI/UX-Design einer Social-Media-Nachbarschafts-App bestimmte Funktionen und Merkmale berücksichtigen, die für eine erfolgreiche Community-Plattform erforderlich sind. Beispielsweise sollte es einfach sein, Beiträge zu erstellen und zu teilen, auf Kommentare zu antworten und Nachrichten an andere Benutzer zu senden. Es sollte auch Möglichkeiten geben, um Benutzerprofile zu erstellen und zu verwalten sowie umfassende Datenschutz- und Sicherheitsfunktionen zu bieten.

Insgesamt ist das UI/UX-Design einer Social-Media-Nachbarschafts-App von entscheidender Bedeutung für den Erfolg der App und die Zufriedenheit der Benutzer. Es ist wichtig, dass das Design auf die Bedürfnisse und Anforderungen der Benutzer zugeschnitten ist und ein einfaches, intuitives und ansprechendes Nutzungserlebnis bietet.
\setauthor{Martin Hausleitner}
\subsection{Inspiration}
Während des Designprozesses für die App habe ich Recherchen im Bereich App-Design durchgeführt. Mein Ziel war es, die App so intuitiv wie möglich zu gestalten, um eine benutzerfreundliche Erfahrung zu gewährleisten. Dazu habe ich mich an bekannten Social-Media-Apps orientiert, wie zum Beispiel Twitter, Instagram und TikTok, die bereits auf dem Markt sehr erfolgreich sind und von vielen Menschen vertraut genutzt werden.

Darüber hinaus hat auch die erfolgreichste Nachbarschafts-App in Deutschland, "Nebenan", viel zur Grundlage der App beigetragen. Allerdings habe ich festgestellt, dass ihre App sehr kompliziert aufgebaut und unübersichtlich ist, was für uns eine Chance darstellte, es besser zu machen.

Zusätzlich haben wir uns von Websites wie Dribble und Mobbin inspirieren lassen, die uns geholfen haben, ein einfaches und schlichtes Design für die App zu entwickeln. Insgesamt war die Recherche und Inspiration für das Design der App ein wichtiger Schritt, um sicherzustellen, dass unsere Benutzer eine ansprechende und intuitive Erfahrung haben.

\subsection{Prototyping}
Prototyping ist ein wichtiger Schritt bei der Entwicklung von mobilen Apps. Es ermöglicht Entwicklern, Designern und anderen Stakeholdern, eine frühzeitige Vorstellung davon zu bekommen, wie die App funktionieren wird und wie sie aussehen wird. Durch die Erstellung eines Prototyps können auch Fehler im Design und in der Funktionalität identifiziert werden, bevor die App in die eigentliche Entwicklung geht.

Es gibt verschiedene Arten von Prototypen, darunter Low-Fidelity- und High-Fidelity-Prototypen. Low-Fidelity-Prototypen sind einfache Skizzen oder Wireframes, die nur die grundlegenden Funktionen und das Layout der App abbilden. High-Fidelity-Prototypen sind detaillierter und können interaktive Funktionen und Designelemente enthalten.

Es gibt viele Tools und Plattformen, die zur Erstellung von Prototypen verwendet werden können. Ein Beispiel ist Adobe XD, das es Entwicklern und Designern ermöglicht, schnelle und einfache Prototypen zu erstellen. Andere Tools wie Sketch, Figma oder InVision sind ebenfalls beliebt.

Der Vorteil von Prototyping ist, dass es Entwicklern und Designern ermöglicht, frühzeitig Feedback von Benutzern zu erhalten und Änderungen vorzunehmen, bevor die App tatsächlich entwickelt wird. Außerdem können Entwickler und Designer Zeit und Ressourcen sparen, indem sie sich auf die richtigen Funktionen und das richtige Design konzentrieren und unnötige Funktionen und Designs vermeiden.

Im Fall von Flutter als Entwicklungsplattform für die App ist das Prototyping ebenfalls wichtig. Flutter bietet verschiedene Widgets und Design-Tools, mit denen Entwickler schnell und einfach Designelemente erstellen und implementieren können. Durch ein gut durchdachtes Prototyping kann Zeit und Aufwand beim Entwickeln gespart werden, da Design-Entscheidungen bereits getroffen wurden und Entwickler sich auf die Umsetzung konzentrieren können.

Insgesamt ist Prototyping ein unverzichtbarer Schritt bei der Entwicklung von mobilen Apps. Es ermöglicht Entwicklern und Designern, frühzeitig Feedback von Benutzern zu erhalten und Änderungen vorzunehmen, bevor die App tatsächlich entwickelt wird. Durch die Verwendung von Tools wie Adobe XD, Sketch oder Figma können Prototypen schnell und einfach erstellt werden, was Zeit und Ressourcen spart. Prototyping ist auch für die Verwendung von Flutter als Entwicklungsplattform wichtig, da es Entwicklern ermöglicht, Zeit und Aufwand zu sparen und sich auf die Umsetzung zu konzentrieren.
\subsubsection{Framer}

Framer ist eine Software, die es Benutzern ermöglicht, schnell und einfach ansprechende Prototypen von mobilen Anwendungen und Websites zu erstellen. Das Tool wurde ursprünglich als Prototyping-Tool für Designer entwickelt, um Designs schnell zu testen und zu verfeinern, bevor sie in die Entwicklung übergehen.

Erfolgreiche Apps wie Spotify haben Framer im Rahmen ihres
Prototyping-Prozesses verwendet, um schnell und effizient
funktionierende App-Designs zu erstellen. Framer ist eine
schnelle und effiziente Möglichkeit, um Ideen in die Tat
umzusetzen, ohne sich durch langwierige Entwicklungsprozesse
zu quälen.

\includegraphics[width=1\textwidth]{pics/nochba-framer-prototype-screenshot.png}


Framer wurde während des Hackerthon Linz hackt verwendet, um den ersten Prototypen zu erstellen. Die Wahl von Framer war aufgrund seiner Geschwindigkeit und Effizienz in der Erstellung von funktionsfähigen App-Designs und der Erfahrung, die der Benutzer bereits mit dem Tool hatte, getroffen worden.

Seit der Erstellung des ersten Prototyps hat sich das Geschäftsmodell von Framer jedoch geändert. Es ist jetzt ein Website-Baukasten, der es Benutzern ermöglicht, einfach und schnell ansprechende Websites zu erstellen, ohne Kenntnisse in der Webentwicklung zu benötigen. Obwohl es nun als Website-Baukasten fungiert, behält Framer immer noch einige seiner Kernfunktionen als Prototyping-Tool bei, was es zu einer guten Option für Designer und Entwickler macht, die schnell Prototypen erstellen möchten.

Insgesamt hat Framer gezeigt, dass es ein schnelles und effektives Tool ist, um Designideen in die Tat umzusetzen. Obwohl es nun als Website-Baukasten fungiert, ist es immer noch eine nützliche Option für Designer und Entwickler, die schnell und einfach Prototypen erstellen möchten.

\subsubsection{Adobe XD}
Als ich zum ersten Mal mit Framer arbeitete, war ich begeistert von den vielen Features und Möglichkeiten, die es bietet. Aber im Laufe der Zeit stellte ich fest, dass es für mein Projekt zu aufwändig war und ich nach einer einfacheren Lösung suchte. So bin ich auf Adobe XD umgestiegen, ein Tool, mit dem ich schon jahrelange Erfahrung hatte.

Obwohl Adobe XD nicht so viele Features wie Framer bietet,
ist es aufgrund seiner Einfachheit und
Benutzerfreundlichkeit ein ideales Tool, um schnell einen
Prototypen zu erstellen. Ich hatte alle Hauptscreens
fertig gestaltet, bevor wir mit der Entwicklung mit Flutter
begannen. Die anderen wichtigen Screens habe ich dann
später gestaltet, als ich mit Flutter vertrauter war und
besser abschätzen konnte, wie aufwändig es in Flutter
umzusetzen war.

\includegraphics[width=0.95\textwidth]{pics/nochba-adobe-xd-protoype-screenshot.png}


Wie man auf dem obigen Bild sehen kann, ist unser Prototyp
eine Sammlung verschiedener Designs und Evolutionen des
Designs. Adobe XD war für unser Projekt perfekt, da ich der
alleinige Designer war und somit keine
Kollaborationsfunktion benötigte. Allerdings, da ich jetzt
mit dem bekanntesten Designtool Figma vertraut bin und es billiger ist, wenn mehrere
Personen an einem Design arbeiten und auch mehr Funktionen
bietet, würde ich aktuell mit Figma weiterarbeiten.

Zusammenfassend lässt sich sagen, dass Adobe XD ein mächtiges Tool für UI/UX-Designer ist, das einfach zu bedienen und ideal für kleine bis mittelgroße Projekte ist. Es bietet zwar nicht so viele Funktionen wie andere Tools, ist aber für schnelle Prototypenerstellung und einfache Zusammenarbeit mit Entwicklern und Stakeholdern sehr gut geeignet.

\subsection{Design}
allg.

\subsubsection{Design System}
bild von allen componetns

\subsubsection{Farben}
Die Farbpalette einer Marke ist ein wichtiger Teil des UI-Designs und trägt dazu bei, ein einheitliches Erscheinungsbild zu schaffen. Bei der Gestaltung einer Nachbarschafts-App haben wir uns verschiedene Farbpaletten von anderen Apps angesehen und festgestellt, dass die meisten großen Apps grün gestaltet sind.

Obwohl Grün oft mit Natur und Gemeinschaft in Verbindung gebracht wird, wollten wir uns von diesen etablierten Konventionen abheben. Wir haben uns für Orange entschieden, da es auffällig und ungewöhnlich ist und somit das Potenzial hat, unsere App von anderen Nachbarschafts-Apps zu unterscheiden.

Darüber hinaus passt Orange gut zu unseren Werten, da es für
Wärme, Freundschaft und Optimismus steht - Eigenschaften,
die wir in unserer App fördern möchten. Wir hoffen, dass
unsere Farbauswahl dazu beiträgt, dass unsere Nutzer sich in
unserer App wohl und willkommen fühlen und die Farbe sich
als wiedererkennbares Markenzeichen etabliert.



\includegraphics[width=1\textwidth]{pics/colors.png}

Bevor wir uns für unseren finalen Farbton entschieden haben, haben wir eine Vielzahl von verschiedenen orangen Farbtönen für unseren Prototypen ausprobiert und sie dann für ein paar Tage beobachtet, um ein Gefühl für die Farbe zu bekommen.

Schließlich haben wir uns für die endgültigen Farben entschieden, wie in der Abbildung dargestellt.

Unsere Hauptcontainerfarbe für die meisten Ansichten ist Weiß, ebenso wie die Hintergrundfarbe der Posts. Allerdings mussten wir eine Farbe finden, die gut als Hintergrund passt. Hier haben wir ein helles Grau verwendet.

Für unsere Textfarbpalette haben wir fast Schwarz verwendet. Dies wurde bewusst so gewählt, da Schwarz dadurch weicher wirkt. Um Untertitel weniger präsent zu gestalten, haben wir einen etwas helleren Grauton gewählt.
\subsubsection{Icons}
\includegraphics[width=1\textwidth]{pics/icons.png}
In der Abbildung oben sind alle verwendeten Icons in unserer App zu sehen. Es war uns ein wichtiges Anliegen, die App mit so vielen Icons wie möglich auszustatten, um die Bedienung intuitiver und benutzerfreundlicher zu gestalten. Nachdem wir uns mehrere Icon-Packs angesehen hatten, haben wir uns hauptsächlich für die frei verfügbaren Remixicons entschieden. Uns war wichtig, ein eher unbekanntes Icon-Pack zu verwenden, um unsere App von anderen abzuheben und ihr ein einzigartiges Aussehen zu geben. Besonders das verspielte, runde Design hat uns gut gefallen. Ein weiterer großer Vorteil von Remixicons ist, dass es ein Flutter-Paket gibt, was die Implementierung in Flutter vereinfacht hat.

Obwohl Remixicons über 2.271 Icons verfügt, haben uns bei einigen Icons die Material Icons von Google besser gefallen. Daher haben wir einige Icons von Google Material Icons verwendet, insbesondere die abgerundeten Icons, um das Design insgesamt nicht zu hart erscheinen zu lassen. Als ich kein passendes Icon finden konnte, um den Abstand zwischen zwei Nachbarn zu symbolisieren, habe ich mich selbst daran gesetzt, ein passendes Icon im Stil der anderen zu entwerfen.
\subsubsection{Fonts}
fotos von fonts
welche fonts
wie ist due typrographie aufgebaut
\subsubsection{Logo}

\includegraphics[width=0.95\textwidth]{pics/final-logo.png}

% \includegraphics[width=0.35\textwidth]{pics/app-logo.png}

% überschrift logo historie
\paragraph{Logo Historie}



\includegraphics[width=0.95\textwidth]{pics/logo-historie.png}



Im Zuge des Designprozesses haben wir hohe Ansprüche an das
Logo unserer App gestellt, da das logo unser Projekt repräsentiert. Es war uns wichtig, dass es einfach gehalten
und leicht erkennbar ist. Wir haben uns Zeit gelassen, um
den endgültigen Entwurf des Logos zu finalisieren, da uns
die vorherigen Entwürfe nicht vollständig zufrieden
stellten. Ich habe gezielt nach anderen Logos gesucht, die
eine Verbindung zu Nachbarschaften oder Häusern herstellen
und Ideen auf unserem Discord-Server gespeichert. Es sollte
immer ein Haus erkennbar sein, da Häuser schnell mit
Nachbarschaften in Verbindung gebracht werden.

In den früheren Versionen der App hatte ich Schwierigkeiten, ein ansprechendes App-Icon zu gestalten, da ich kein eigenständiges Icon hatte. Aus diesem Grund habe ich mich dazu entschlossen, den Schriftzug und das App-Icon (Logo) separat zu gestalten, wie es auch bei anderen Apps üblich ist. Das endgültige Logo habe ich dann erst Ende 2022 entworfen, da mir alle anderen Designs nicht gefielen. Es ist ein Haus mit einer sprechenden Sprechblase, die lächelt, geworden. Dies soll symbolisieren, dass man innerhalb des Hauses sprechen kann und das Lächeln soll verdeutlichen, dass man Freude mit seinen Nachbarn teilen kann. Generell hat ein Lächeln immer eine positive Wirkung auf den Betrachter und es verdeutlicht, dass es sich bei unserer App um eine Anwendung für den Austausch unter Menschen handelt.

Wir haben bewusst eine runde, verspieltere Schriftart gewählt, da sie besser mit der Farbmischung harmoniert als die vorherige Schriftart.
\subsection{App Design}
\subsubsection{Thumb Zone Prinzip}
\includegraphics[width=0.3\textwidth]{pics/edit-name-screen.png}
Wir haben versucht, das Thumb Zone Prinzip in unserem Design-Muster zu implementieren, indem wir wichtige UI-Elemente immer im unteren Bereich der App platziert haben. Wenn man die Abbildung betrachtet, kann man deutlich erkennen, dass wir das Layout für die Namensbearbeitung extra unten platziert haben, um alle Klick-Elemente bequem mit dem Daumen bedienen zu können. Wir haben auch alle wichtigen Buttons wie "Speichern" in unserer primären Farbe gestaltet und rechts ausgerichtet, um einen leichteren Zugriff zu gewährleisten. Außerdem haben wir es bei unserer App so implementiert, dass man das Textfeld nicht durch Klicken ändern muss, sondern einfach die Enter-Taste auf der Tastatur drücken kann, um das Eintippen von Daten zu erleichtern. Unwichtige Buttons haben wir in grau gestaltet, wie z.B. den X-Button in der Abbildung. Ein kleines, aber nützliches Feature, das wir eingebaut haben, ist der Löschen-Button neben dem Textfeld, der es dem Benutzer ermöglicht, den gesamten Text bequem zu löschen. Leider ist uns gegen Ende die Zeit ausgegangen, um diese Prinzipien vollständig umzusetzen. In zukünftigen Versionen der App möchten wir sicherstellen, dass diese Prinzipien einheitlich angewendet werden.
\subsubsection{Design Historie}

\begin{figure}[h]
  \centering
  \includegraphics[width=1\textwidth]{pics/app-design-history.png}
  \caption{Screenshots verschiedener Versionen des App-Home-Screen}
  \label{fig:app-design-history}
\end{figure}
Im Rahmen des Hackathons Linz haCKt wurde das erste Design in Abbildung \ref{fig:app-design-history} ganz links erstellt. Hierbei wurden mehrere Brainstorming-Meetings mit Mentoren abgehalten, um ein Grundkonzept in Adobe XD zu gestalten. Anschließend wurden die groben Züge des Designs in Absprache mit dem Team festgelegt. Ein funktionsfähiger Prototyp wurde in Framer erstellt und am Tag der Endpräsentationen vorgestellt und verlinkt. Nach Linz haCKt wurde das zweite Design in Framer entwickelt.

Zwischen dem zweiten Framer-Design\ref{fig:app-design-history} und dem ersten Adobe XD-Design\ref{fig:app-design-history} wurde bewusst kein Abstand am Rand eines Beitrags gehalten, um den vollen Handy-Display auszunutzen. Der Filter-Button wurde mit der Suchleiste verbunden, um das Design stimmiger zu gestalten. Bei den Profilinformationen wurde auf Icons umgestellt, um es für neue Nutzer verständlicher zu machen. Diese Entscheidung wurde kurzzeitig verworfen, jedoch später in der Flutter-App wieder eingebaut.

Wie man erkennen kann, wurde für das Finale Adobe
XD-Design\ref{fig:app-design-history} eine orangene Farbe für die Kapitelfarben
gewählt. Der letzte Screenshot zeigt die Flutter-App, wie
sie im Play Store erhältlich ist. Die größte Änderung
gegenüber dem vorherigen Design betrifft den Hintergrund
bei der Suche und der Kategorieauswahl. Der Hintergrund
wurde hellgrau gemacht, um die Beiträge besser hervorzuheben.
Bei der Entwicklung mit Flutter stellte sich heraus, dass
ein Strich über dem aktiven Icon sehr aufwändig zu
implementieren ist, weshalb darauf verzichtet wurde.



\section{Website}
\setauthor{Martin Hausleitner}
\begin{figure}[h]
  % \centering
  \includegraphics[width=1\textwidth]{pics/website-design.png}
  \caption{Screenshot der Nochba Website: nochba.at}
\end{figure}

Aufgrund der Medienpräsenz, die unser Projekt durch unsere Teilnahme an Linz hACkT erfahren hat und der Tatsache, dass wir keinen zentralen Anlaufpunkt für Informationen hatten, haben wir uns dazu entschlossen, eine schlichte Landing Page mit grundlegenden Informationen unter \href{https://nochba.at}{nochba.at} zu gestalten.

Unsere Teilnahme an dem mPreneur Social Mobile Entrepreneurship von Arsham Edalatkhah hat uns zu einer internationalen Aufmerksamkeit verholfen. Dies hat uns dazu bewogen, eine englischsprachige Website unter \href{https://nochba.com}{nochba.com} zu erstellen, um unsere Arbeit einem breiteren Publikum zugänglich zu machen und unsere Reichweite zu erhöhen.


\subsection{Beta-tester anmeldung}
Das wichtigste Merkmal der Website ist die Option für Nutzer, sich für die Testphase zu registrieren. Ursprünglich wurde versucht, die E-Mail-Adressen der Nutzer über die Google Sheets API zu speichern. Allerdings erwies sich dieser Ansatz als ineffektiv und dauerte mehrere Stunden. Als Alternative wurde das Framer-Add-On von Mailchimp.com genutzt, um Zeit zu sparen. Der kostenlose Plan von Mailchimp war für die Bedürfnisse ausreichend. Stand 7.3.2023 wurden 34 Beta-Anmeldungen über das Formularfeld gesammelt.


\subsection{Design}
Für die Gestaltung der Landing Page ließ sich der Designer
von den Vorlagen von Framer inspirieren und nutzte
vorgefertigte Abschnitte. Diese wurden jedoch so
modifiziert, dass sie dem Designsysten gerecht werden. Die
Farbpalette wurde beibehalten und es wurde darauf geachtet,
dass die Website möglichst einfach gestaltet ist.

\subsection{Kontent}
Auf der Website sind folgende Informationen abgebildet:
\begin{itemize}
  \item Zeitungsartikel
        \begin{itemize}
          \item \href{https://www.meinbezirk.at/linz/c-wirtschaft/linzer-nachwuchs-hacker-beweisen-sich-in-nordmazedonien_a5654095}{Mein Bezirk}
          \item \href{https://www.linz.at/medienservice/2022/202203_114704.php}{Linz}
          \item \href{https://www.nachrichten.at/oberoesterreich/linz/von-linz-nach-nordmazedonien;art66,3728292}{OÖNachrichten}
          \item \href{https://jugendhackt.org/video/nochba/}{DorfTV}
          \item Kronen Zeitung (nur auf Papier)
        \end{itemize}
  \item Unsere Mission
  \item Beitrag Kategorien
  \item Auszeichnungen
        \begin{itemize}
          \item Immotopia Innovation Award
          \item mPreneur Austria
          \item Linz hACkt
        \end{itemize}
  \item Features der App
  \item Partner der Diplomarbeit
  \item Über uns
  \item Links
        \begin{itemize}
          \item \href{https://github.com/Martin-Hausleitner/Nochba}{Github Repo}
          \item \href{https://www.instagram.com/nochba.at/}{Instagram}
          \item Email \href{mailto:project@nochba.com}{project@nochba.com}
        \end{itemize}
\end{itemize}


\subsection{Hosting}
Die Website wurde mithilfe von \href{https://framer.com}{Framer} erstellt und auf deren
Hosting-Plattform gehostet. Um die Domain-Namen zu nutzen,
die über GoDaddy (nochba.at und nochba.com) gekauft wurden, wurde eine Weiterleitung
mit Maskierung auf die entsprechende Framer-URL
eingerichtet. Durch diese Maßnahme wurde im Browser bei der
URL die gekaufte Domain angezeigt.

\subsubsection{SSL-Zertifikat}
Leider hat die Website derzeit kein SSL-Zertifikat, da durch
die Maskierung das Framer SSL-Zertifikat verloren geht.
Allerdings wird die Website im Browser nicht als bedrohlich
angezeigt, weshalb wir uns entschieden haben, kein
SSL-Zertifikat zu erstellen, um den Aufwand zu minimieren.

\section{Backend}

\subsection{Firebase}
\setauthor{Martin Hausleitner}
Firebase ist ein BaaS, das uns Entwicklern eine erhebliche Menge an Stress abnimmt. Das Hosting von Datenbanken oder in unserem Fall Cloud-Funktionen funktioniert mit nur wenigen Klicks. Wir müssen uns keine Gedanken über Skalierung oder Ausfälle machen. Unser Firebase-Backend hosten wir auf der Google Cloud in Frankfurt (EUR3 Europe-West), um eine geringe Latenz zu gewährleisten. Zu Beginn haben wir den kostenlosen Plan "Spark" genutzt, was für uns ausreichend war, da wir noch nicht die Firebase-Cloud-Funktionen genutzt haben. Um Geld zu sparen, haben wir dann lange Zeit mit dem lokalen Firebase-Emulator an den Cloud-Funktionen gearbeitet. Im Januar 2023 sind wir dann auf den "Blaze"-Plan umgestiegen, um die Firebase-Cloud-Funktionen nutzen zu können. Bis zum Stand März 2022 haben wir nur wenige Euro für den Verbrauch gezahlt. Wir waren auch sehr beeindruckt von der exzellenten Dokumentation und den vielen Tutorials auf YouTube, die Firebase anbietet.


\subsection{Firebase SDKs}
\setauthor{Sandin Habibovic}
Firebase SDK ist eine Sammlung von Software Development Kits, die von Firebase bereitgestellt werden, um Cloud-basierte Anwendungen zu erstellen.
Firebase SDK ist für eine Vielzahl von Programmiersprachen verfügbar, darunter JavaScript, Swift, Kotlin, Java und unter anderem Dart von Flutter.
Die Firebase SDK bietet eine breite Palette von Funktionen, mit denen leistungsstarke Anwendungen erstellt werden können, darunter Authentifizierung, Cloud Messaging, Cloud Firestore, Realtime Database, Cloud Functions. Mit diesen Funktionen können schnell und einfach Funktionen wie Benutzerverwaltung, Datenverwaltung, Messaging und Benachrichtigungen in Anwendungen integriert werden. Außerdem gibt es zu Firebase SDK eine umfangreiche Dokumentation und Support-Tools.

\subsection{Firebase Authentication}
\setauthor{Sandin Habibovic}
Email und Passwort authentication

\subsection{Cloud Firestore}
\setauthor{Sandin Habibovic}
NoSQL-Database, Dokument-basierte Speicherung, Subcollections
\subsection{Cloud Storage}
\setauthor{Sandin Habibovic}
Speicherung von Profilbilder und Post-Bilder
\subsection{Firebase Cloud Functions}
\setauthor{Martin Hausleitner}
Firebase Cloud Functions sind eine großartige Möglichkeit,
um Business-Logik abzubilden. Es handelt sich um serverlose
Funktionen, die in JavaScript oder TypeScript geschrieben
werden und einzeln auf der Google Cloud gehostet werden. Je
nach Nachfrage werden sie automatisch skaliert oder komplett
abgeschaltet. Ein Nachteil ist, dass die Startup-Zeit länger
sein kann als bei einem herkömmlichen Backend, das immer
läuft. Allerdings kann man durch Cloud Functions viel Geld
sparen, da nur die Prozessorlaufzeit bezahlt werden muss.

Firebase Cloud Functions ermöglichen es Entwicklern, auf verschiedene Ereignisse in Firebase-Produkten zu reagieren. Zum Beispiel, wenn sich Daten in der Firestore-Datenbank ändern oder ein neuer Nutzer in Firebase Authentication registriert wird. Wenn ein solches Ereignis eintritt, wird die entsprechende Cloud Function automatisch ausgeführt.

Wir haben alle Cloud Functions in TypeScript programmiert,
da TypeScript Typsicherheit und erweiterte Fehlererkennung
bietet.


\subsubsection{Regestrierung mit einen Verifizierungscode}
\setauthor{Martin Hausleitner}

Die Cloud-Funktion "checkVerificationCode" wird ausgeführt, wenn ein Nutzer sich mit einem Verifizierungscode registrieren möchte Wenn der Benutzer nicht authentifiziert ist, wird eine HttpsError ausgelöst. Dann wird der Verifizierungscode überprüft, um sicherzustellen, dass er die korrekte Formatierung hat. Wenn der Code ungültig ist, wird eine weitere HttpsError ausgelöst.

Als Nächstes wird der Verifizierungscode mit der Datenbank
abgeglichen, um sicherzustellen, dass er aktiv und noch
nicht zu oft verwendet wurde. Wenn der Code erfolgreich
validiert wird, wird die Adresse des Benutzers abgerufen und
deren Koordinaten mithilfe von einer API call
anOpenStreetMap mit der Funktion "getOSMCoordinatesFromAddress"
ermittelt. Dann wird die Entfernung zwischen der Adresse des
Benutzers und der Adresse, die dem Verifizierungscode
zugeordnet ist, berechnet mit der Funktion "getDistanceFromLatLonInMeters". Wenn die Entfernung nicht
innerhalb des zulässigen Bereichs liegt, wird eine weitere
HttpsError ausgelöst.

Schließlich werden die Informationen des Benutzers und des
Verifizierungscodes in der Datenbank aktualisiert, um
anzuzeigen, dass der Benutzer erfolgreich verifiziert wurde.
Die Cloud-Function gibt true zurück, um anzuzeigen, dass die
Verifizierung erfolgreich war.

In jeder Phase der Funktion wird ein Logger verwendet, um Informationen über den Status der Funktion zu protokollieren.

\subsubsection{Regestrierung mit Gerät Koordinaten}
\setauthor{Martin Hausleitner}

Die Cloud-Funktion "checkAddressWithDeviceLocation" erfordert eine authentifizierte Anfrage und
erhält eine Adresse, eine Längen- und Breitengradkoordinate
vom Gerät des Benutzers. Die Funktion prüft, ob alle
erforderlichen Daten vorhanden sind und ruft dann die
Funktion "getOSMCoordinatesFromAddress" auf, um die
Koordinaten der angegebenen Adresse zu erhalten. Es wird
auch die Funktion "getDistanceFromLatLonInMeters"
aufgerufen, um die Entfernung zwischen der Adresse und den
Koordinaten des Geräts des Benutzers zu berechnen.

Wenn die Entfernung größer ist als ein vordefinierter maximaler Abstand, wird eine Fehlermeldung ausgegeben und die Funktion gibt "false" zurück. Andernfalls speichert die Funktion die Koordinaten der Adresse und die Entfernung zwischen den Koordinaten des Geräts des Benutzers und der Adresse in der Firestore-Datenbank. Die Funktion ruft auch die Funktion "getOSMSuburbFromCoords" auf, um den Vorort der Adresse zu erhalten, und speichert diesen ebenfalls in der Firestore-Datenbank.

Die Funktion gibt "true" zurück, wenn die Verifizierung
erfolgreich abgeschlossen ist.

In jeder Phase der Funktion wird ein Logger verwendet, um Informationen über den Status der Funktion zu protokollieren.

\subsubsection{Verifizierungscode generieren}
\setauthor{Martin Hausleitner}
Die Cloud-Funktion "generateVerificationCode" definiert Konstanten
für das Intervall zwischen der Generierung von Codes, die
maximale Anzahl von Codes und die Reichweite in Metern.

Anschließend wird der letzte code des nutzers aus der
Datenbank geholt, um zu überprüfen, ob seit dem
letzten generierten Code ausreichend Zeit vergangen ist.
Wenn ja, wird der zuletzt generierte Code zurückgegeben.

Wenn nicht, wird eine Schleife gestartet, um einen neuen
Code zu generieren mit der Funktion "generateRandomVerificationCode". Der generierte Code wird dann mit
Firestore abgeglichen, um sicherzustellen, dass er nicht
bereits verwendet wurde.

Wenn der generierte Code eindeutig ist, wird überprüft ob
der Benutzer verifiziert ist. Wenn dies nicht der Fall ist,
wird ein Fehler zurückgegeben. Andernfalls wird der
generierte Code in die
Firestore-Datenbank eingefügt um den letzten generierten
Code und das Datum der Generierung zu speichern.

Das Skript gibt dann den generierten Code zurück. In jeder
Phase der Funktion wird ein Logger verwendet, um
Informationen über den Status der Funktion zu
protokollieren.

\subsubsection{Koordienaten von einer Adresse bestimmen}
\setauthor{Martin Hausleitner}
Ein wichtiger Teil für die Verifizierung ist, dass wir die Adresse des Nutzers in Koordinaten umwandeln, damit wir im Backend damit arbeiten können. Zunächst haben wir die Google Geocode API genutzt, allerdings kostet diese Geld. Wir haben eine bessere Option gefunden, die besser zu unseren Projekt-Philosophie passt: nämlich Nominatim, eine Open-Source-Geocoding-API für OpenStreetMap-Daten. Nominatim bietet eine kostenlose API für genau unser Problem an.

Die Funktion \textbf{getOSMCoordinatesFromAddress} nutzt die Bibliothek \textbf{axios}, um eine Anfrage an die Nominatim-API zu senden. Die API wird genutzt, um Koordinaten für eine Adresse zu erhalten.

Die Funktion nimmt einen Parameter \textbf{address} vom Typ \textbf{string} entgegen, welcher die Adresse enthält, für die Koordinaten abgerufen werden sollen.

Mithilfe von \textbf{axios.get} wird eine HTTP GET-Anfrage an die Nominatim-API gesendet. Die Adresse wird als URL-Parameter im Format \textbf{q=Adresse} übergeben. Die API liefert die Koordinaten als JSON-Objekt zurück, welches in der Variable \textbf{data} gespeichert wird.

Die Funktion überprüft dann, ob die Anfrage ein Ergebnis zurückgeliefert hat. Falls nicht, wird eine Fehlermeldung mit der Adresse ausgegeben.

Wenn ein Ergebnis vorhanden ist, wird das erste Ergebnis (in der Variable \textbf{firstResult}) verwendet, um eine neue Instanz der \textbf{GeoPoint}-Klasse zu erstellen. Diese Klasse ist Teil der Firebase-Admin-Bibliothek und ermöglicht die Speicherung von geografischen Koordinaten in einer Firestore-Datenbank.

Schließlich gibt die Funktion die neuen Koordinaten als \textbf{GeoPoint}-Objekt zurück, das die Breiten- und Längengrade in Zahlen enthält.

\subsubsection{Entferung von zwei Nutzern berechnen}
\setauthor{Martin Hausleitner}
Die Cloud-Funktion "getDistanceFromTwoUsers" berechnet die
Entfernung zwischen zwei Benutzern. Die Funktion erwartet
eine PostId und einen Authentifizierten bneutzer. Dann wird eine
Überprüfung durchgeführt, ob der Post mit der angegebenen ID
existiert und ob der Post eine gültige Reichweite hat. Es
werden auch Überprüfungen durchgeführt, ob die
Benutzerkoordinaten vorhanden sind, und ob die Entfernung
zwischen beiden Benutzern innerhalb des Postbereichs liegt.

Die Funktion nutzt die importierten Funktionen
getDistanceFromLatLonInMeters und getNearestDistance, um die
Entfernung in Metern zu berechnen. Allerdings wird die
berechnete Distanz grob gerundet, um die Privatsphäre der
Nutzer zu wahren. Dabei
werden die Längen- und Breitengradkoordinaten von zwei
Benutzern miteinander verglichen, die aus der
Firestore-Datenbank abgerufen werden. Wenn die Entfernung
größer als die Reichweite des Posts ist, wird ein Fehler
ausgegeben.

Schließlich gibt die Funktion die Entfernung zurück.

\subsubsection{Entfernung von zwei geographischen Koordinaten berechnen}
\setauthor{Martin Hausleitner}

\begin{lstlisting}[language=Java,caption=getDistanceFromLatLonInMeters Funktion]
    export function getDistanceFromLatLonInMeters(
        lat1: number,
        lon1: number,
        lat2: number,
        lon2: number
      ) {
        if (lat1 < -90 || lat1 > 90 || lon1 < -180 || lon1 > 180) {
          throw new Error(
            "Invalid coordinate: lat1 must be between -90 and 90, lon1 must be between -180 and 180"
          );
        }
        if (lat2 < -90 || lat2 > 90 || lon2 < -180 || lon2 > 180) {
          throw new Error(
            "Invalid coordinate: lat2 must be between -90 and 90, lon2 must be between -180 and 180"
          );
        }
        const R = 6371; // Radius der Erde in km
        const dLat = deg2rad(lat2 - lat1); // deg2rad unten
        const dLon = deg2rad(lon2 - lon1);
        const a =
          Math.sin(dLat / 2) * Math.sin(dLat / 2) +
          Math.cos(deg2rad(lat1)) *
            Math.cos(deg2rad(lat2)) *
            Math.sin(dLon / 2) *
            Math.sin(dLon / 2);
        const c = 2 * Math.atan2(Math.sqrt(a), Math.sqrt(1 - a));
        const d = R * c * 1000; // Distanz in Metern
      
        if (lat1 === lat2 && lon1 === lon2) return 0;
        return d;
      }
      
      function deg2rad(deg: number) {
        return deg * (Math.PI / 180);
      }
      
\end{lstlisting}
Der vorliegende Code implementiert eine Funktion, die die Distanz in Metern zwischen zwei geographischen Koordinaten (Breitengrad und Längengrad) auf der Erdoberfläche berechnet. Die Funktion verwendet die Haversine-Formel, die auf der Kugelgeometrie basiert, um die kürzeste Entfernung zwischen zwei Punkten auf der Erdoberfläche zu berechnen.

Die Funktion "getDistanceFromLatLonInMeters" hat vier Parameter: "lat1" und "lon1" sind die Breiten- und Längengrade des ersten Punktes, während "lat2" und "lon2" die Breiten- und Längengrade des zweiten Punktes sind, zwischen denen die Distanz berechnet werden soll.

Zu Beginn des Codes werden die Eingabeparameter auf ihre Gültigkeit geprüft und eine Fehlermeldung wird ausgegeben, falls eine der Koordinaten außerhalb des Bereichs von -90 bis 90 für die Breite und -180 bis 180 für die Länge liegt.

Die Funktion berechnet dann die Differenzen der Breiten- und Längengrade sowie den Radius der Erde (R) in Kilometern. Die Differenzen werden dann in Radianten umgewandelt und die Haversine-Formel wird angewendet, um die Entfernung in Kilometern zu berechnen. Schließlich wird das Ergebnis in Meter umgewandelt und zurückgegeben.

Die Funktion "deg2rad" wird als Hilfsfunktion definiert, um Grad in Radianten umzurechnen.

Die Funktion gibt 0 zurück, falls die beiden
Eingabeparameter denselben Wert haben, um zu vermeiden, dass
eine sehr kleine Distanz als Ergebnis ausgegeben wird, wenn
es sich um denselben Punkt handelt.

Quelle: https://www.movable-type.co.uk/scripts/latlong.html

\subsubsection{Bestimmung der nächstgelegenen Entfernung}
\setauthor{Martin Hausleitner}

Um die Privatsphäre der Nachbarn zu gewährleisten, benötigen wir eine Funktion, die uns den nächstgelegenen Abstand zu den Nachbarn gibt, ohne den genauen Abstand preiszugeben. Die Funktion heißt "getNearestDistance" und bekommt einen Meterwert als Parameter.

Die Funktion erstellt ein Array mit den Optionen [100, 200, 500, 1000, 5000, 10000, 15000] und setzt die Variable "nearest" auf den ersten Wert im Array.

Dann wird eine Schleife ausgeführt, die durch jedes Element im Array "options" geht und prüft, welches Element am nächsten zum angegebenen Abstand "distance" liegt. Wenn ein Element näher ist als das bisher am nächsten liegende Element, wird "nearest" aktualisiert.

Schließlich wird überprüft, ob "nearest" größer oder gleich
1000 ist, und je nachdem wird der Abstand entweder in
Kilometern oder Metern zurückgegeben. Wenn der Abstand
größer oder gleich 1000 ist, wird die Einheit "km"
hinzugefügt, ansonsten wird "m" hinzugefügt. Wir haben
absichtlich keine Switches oder If-Bedingungen benutzt, da
wir es jetzt mit der aktuellen Funktion schneller schaffen,
die Abstände zu ändern. Wir evaluieren noch, wie das Array
mit den Optionen aussieht.


\subsubsection{Nachbarschaft von Nutzer bestimmen}
\setauthor{Martin Hausleitner}
Um unseren App-Nutzern ein besseres Verständnis für die Nachbarschaften zu geben, in denen ihre Nachbarn wohnen, zeigen wir bei jedem Benutzer die Nachbarschaft an, in der sie leben. Diese Information wird automatisch in der Datenbank gespeichert, wenn der Nutzer verifiziert wird. Um diese Funktion zu ermöglichen, verwenden wir die folgende Funktion, die die geografischen Koordinaten des Benutzers verwendet, um die entsprechende Nachbarschaft mithilfe der OpenStreetMap-API abzurufen:

Die Funktion heißt "getOSMSuburbFromCoords" und nimmt zwei Parameter entgegen: "lat" für die geografische Breite und "lon" für die geografische Länge des Benutzers. Diese Funktion gibt eine Promise zurück, die eine Zeichenfolge (String) mit dem Namen der Nachbarschaft des Benutzers enthält. Die Funktion verwendet die "axios"-Bibliothek, um eine GET-Anfrage an die OpenStreetMap-API zu senden. Diese Anfrage enthält die geografischen Koordinaten des Benutzers und die gewünschte Zoomstufe (18), um die Nachbarschaft zu finden. Wenn die Anfrage erfolgreich ist, gibt die Funktion den Namen der Nachbarschaft zurück, der aus den Daten extrahiert wird, die von der API zurückgegeben werden. Wenn der Name der Nachbarschaft nicht verfügbar ist, gibt die Funktion den Namen der Stadt oder der Gemeinde zurück, in der sich der Benutzer befindet. Wenn auch diese Informationen nicht verfügbar sind, gibt die Funktion null zurück. Wenn bei der Anfrage ein Fehler auftritt, wird eine Fehlermeldung ausgelöst.

\subsubsection{Verifizierungscode format überprüfen}
\setauthor{Martin Hausleitner}
Um die Laufzeit bei der Verifizierung von Cloud-Funktionen zu optimieren, ist es sinnvoll, am Anfang der Verifizierung eine Überprüfung durchzuführen, ob der Verifizierungscode das richtige Format hat. Dazu wird die Funktion \verb|verifyVerificationCode| genutzt werden, welche einen \verb|string| als Parameter erwartet und einen \verb|Boolean|-Wert zurückgibt. In der Funktion wird ein regulärer Ausdruck (Regex) definiert, um sicherzustellen, dass der Verifizierungscode den Anforderungen entspricht. Der Regex lautet \verb|/^[a-zA-Z0-9]{10}$/|, was bedeutet, dass der Code aus genau 10 alphanumerischen Zeichen bestehen muss. Mit der Methode \verb|test| wird der übergebene Code auf Übereinstimmung mit dem Regex geprüft und das Ergebnis als \verb|Boolean|-Wert zurückgegeben.


\subsubsection{Verifizierungscode generator}
\setauthor{Martin Hausleitner}

Die Funktion \texttt{generateRandomVerificationCode} erzeugt einen zufälligen Code mit einer Länge von 10 Zeichen, indem sie eine Kombination aus Groß- und Kleinbuchstaben des englischen Alphabets sowie Ziffern von 0 bis 9 verwendet. Dabei wird \texttt{Math.random()} zur Generierung einer Zufallszahl zwischen 0 und 1 genutzt und mit der Länge des Zeichenfolgen-Arrays multipliziert, um eine zufällige Position innerhalb des Arrays auszuwählen. Anschließend wird das ausgewählte Zeichen an das Ergebnis angehängt und dieser Schritt wird für jedes Zeichen wiederholt, bis eine Zeichenkette der Länge 10 generiert wurde.

Diese Funktion wird in der Cloud-Funktion \texttt{generateVerificationCode} verwendet, um einen zufälligen Code zu generieren. Es ist wichtig, dass der Code-Generator in der Lage ist, eine ausreichende Anzahl von Codes zu generieren, damit es keine Kollisionen gibt. Da jeder Code zufällig generiert wird und die Funktion eine zufällige Zeichenkette aus 62 möglichen Zeichen erzeugt, ist es äußerst unwahrscheinlich, dass der gleiche Code zweimal generiert wird. Die Anzahl der möglichen Kombinationen beträgt $62^{10}$, was ungefähr $8.39 \times 10^{17}$ Möglichkeiten entspricht. Daher ist die Wahrscheinlichkeit, dass zwei identische Codes generiert werden, vernachlässigbar.


\subsection{Algolia Search}
\subsection{Algolia SDK}

\subsubsection{Typesense Search}
\subsection{Typesense SDK}
