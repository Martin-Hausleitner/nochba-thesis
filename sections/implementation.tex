\section{Continuous Integration/Delivery}
\subsection{GitHub Actions}

allg. actions warum kosten
\subsubsection{Build IOS}
\author{Martin Hausleitner}

\subsubsection{Build Android}

\subsection{Fastline}
\subsubsection{Build Number increment}
\subsection{Firebase App Distribution}

\section{Mobile Anwendung}
\subsection{Dateistruktur}
\author{Martin Hausleitner}
In Flutter gibt es keine fixe Dateistruktur für eine App,
man kann seine Struktur also selbst überlegen und gestalten.
Im Folgenden beschreibe ich, wie wir unsere Dateistruktur
für eine Flutter-App aufgebaut haben.

In Flutter gibt es keine feste Dateistruktur, stattdessen kann man die Struktur der Dateien und Ordner selbst bestimmen. Für unser Flutter-Projekt haben wir uns für eine Struktur entschieden, die sich an bewährten Praktiken orientiert.

Unsere Dateistruktur sieht wie folgt aus:

\begin{itemize}
    \item \textbf{logic} - Hier befindet sich die Geschäftslogik der App, einschließlich der Firestore-Cloud-Funktionen und Repositories, die API-Aufrufe ausführen.
    \item \textbf{pages} - Hier werden Widgets entworfen, die jeweils eine Seite der App darstellen.
    \item \textbf{routes} - Hier werden die Routen definiert, die tiefere Links ermöglichen.
    \item \textbf{shared} - Hier werden UI-Widgets wie Buttons oder andere Widgets gespeichert, die oft wiederverwendet werden.
    \item \textbf{views} - Hier befinden sich Ansichten, die von mehreren Seiten der App verwendet werden können.
\end{itemize}

Im Nachhinein hätten wir die Dateistruktur anders gestaltet,
z.B. hätten wir das UI als eigenes Package definiert und die
pages und views besser unterteilt.

\subsection{State Management}
enscheidung vergleich warum getx
\author{Martin Hausleitner}

\subsubsection{GetX}


\subsection{Authentifizierung}
% übverschrifft foto Anmelde Flow
\subsubsection{Anmelde Flow}
Diagram
erklärung
screenshots
\subsubsection{Regestrierungs Flow}

Diagram
erklärung
screenshots


\subsubsection{Firebase Authentifizierung}
allg.


\subsection{Feed}
\author{Sandin Habibovic}
foto
aufbau

\subsection{Beiträge}
\author{Sandin Habibovic}
Beiträge sind das Hauptkommunikationsmittel auf der App. Jedem Beitrag muss ein Titel, eine Beschreibung und eine Reichweite, unter der, der Beitrag sichtbar ist, angegeben werden. Weiters ist es möglich einem Beitrag ein Bild und Tags anzuhängen.

\subsubsection{Kategorien}
\author{Sandin Habibovic}
Um Beiträge besser zuordnen zu können, muss der User den Beitrag vor dem Veröffentlichen in eine bestimmte Kategorie einteilen. Diese Kategorien ermöglichen es Usern, die Art Ihrer Anfrage ihm vorhinein besser zu spezifizieren und die Suche nach Beiträgen einer bestimmten Art zu vereinfachen. Bestimmte Kategorien werden weiters in Unterkategorien aufgeteilt, da diese ein zu weit gefächertes Genre an Anfragen umfassen.

Es existieren folgende Kategorien bzw. Unterkategorien:

\begin{compactitem}
    \item Mitteilung
    \begin{compactitem}
        \item Frage
        \item Appell
        \item Warnung
        \item Empfehlung
        \item Gefunden
    \end{compactitem}
    \item Suche
    \begin{compactitem}
        \item Hilfe
        \item Verloren
    \end{compactitem}
    \item Ausleihen
    \item Event
\end{compactitem}



Mitteilung:
Die Kategorie der Mitteilung dient dazu die Nachbarn über ein bestimmtes Ereignis oder Meldung zu informieren oder zu befragen.

Suche:
Die Kategorie der Suche dient dazu mit den Nachbarn im Falle einer Hilfesuche oder eines verloren gegangenen Objekts in Kontakt zu treten.

Ausleihen:
Die Kategorie des Ausleihens dient dazu die Nachbarn nach der Erlaubnis, sich ein bestimmtes Werkzeug oder Objekt ausborgen zu dürfen, zu bitten.

Event:
Die Kategorie des Events dient dazu die Nachbarn auf eine bestimmte Veranstaltung aufmerksam zu machen.


\subsubsection{Tags}
\author{Sandin Habibovic}
Als Tag wird ein Schlüsselwort beschrieben, was man an ein Informationsgut anhängen kann, um es besser beschreiben zu können und/oder besser auffindbar zu machen. In der App werden Tags als eine Erweiterung der Kategorien verwendet, um es Usern zu ermöglichen Ihren Beitrag einem selbstdefinierten Typ zuzuordnen.

\subsubsection{Info}
\author{Sandin Habibovic}
Jeder Beitrag hat eine eigene Sektion, wo wichtige Entscheidungsinformationen angegeben werden, wie der Stadtteil und die ungefähre Entfernung zum gegebenen Nachbarn und das Erstelldatum des Beitrags.

\subsubsection{Kommentare}
\author{Sandin Habibovic}
Die Kommentarfunktion ermöglicht es den Usern unter einem Beitrag Ihre Meinung, Feedback oder sonstiges zu hinterlassen.

\subsubsection{Beitrag oder Kommentar Melden}
\author{Sandin Habibovic}
Um auf unangebrachte Beiträge oder Kommentare schnell reagieren zu können, gibt es die Möglichkeit Beiträge oder Kommentare zu melden. Diese Meldungen werden auf Firestore gespeichert und können dann im Einzelnen überprüft werden. Fürs Melden muss ein Grund ausgewählt und eine genauere Beschreibung angegeben werden.
Gründe fürs Melden eines Beitrags oder Kommentars:

\begin{compactitem}
    \item Unangebrachter Inhalt
    \item Belästigung
    \item Betrug
    \item Spam
    \item Sonstiges
\end{compactitem}

\subsection{Filter}
\author{Sandin Habibovic}
Filtern nach Kategorien
Ordnen nach Likes oder Datum
Absteigend oder aufsteigend sortieren

\subsection{Suche}
\subsection{Typesense}
\subsection{Algolia}
diagram
\subsubsection{Firestore Sync}

\subsubsection{Algolia SDK}


\subsection{Chat}
package genommen warum
\subsubsection{Flyer Package}

\subsection{Profil}
\author{Sandin Habibovic}
Was wird angezeigt?
Name, User Public Info, Posts von User

\subsubsection{Profil Melden}
\author{Sandin Habibovic}

\subsection{Benachrichtigungen}
\author{Sandin Habibovic}
Grund und Funktionsweise von Benachrichtigungen
diagram beschreibung
\subsection{Einstellungen}
\author{Sandin Habibovic}
Email/Passwort ändern, Konto löschen, Sprache einstellen, Benachrichtigungen ein/ausschalten
\subsection{Feedback}
\author{Martin Hausleitner}
feedback feature beschreiben


\section{UI/UX Design}
\author{Martin Hausleitner}
\subsection{Inspiration}
was waren meine vorbilder: nebenan dribble twitter...

\subsection{Prototyping}
\subsubsection{Framer}
fotos von prototyp
erklärung warum framer

\subsubsection{Adobe XD}
fotos von prototyp
beschreibung
warum adobe xd


\subsection{Design}
design patterns
\subsubsection{Farben}
farben history
warum orange
\subsubsection{Icons}
foto von den benutzen icons
welche icons
warum
\subsubsection{Fonts}
fotos von fonts
welche fonts
wie ist due typrographie aufgebaut
\subsubsection{Logo}
logo foto
anforderungen
design ideen
logo history
\subsection{App Design}
eingehen mit fotos und beschreibung auf die haupt screens der app

\subsection{Websiten Design}
2-3 nochba.at design foto
beschreibung

\section{Backend}
\subsection{Cloud}
\author{Martin Hausleitner}

\subsection{Firebase}
\author{Martin Hausleitner}

\subsubsection{Firebase SDKs}
\author{Sandin Habibovic}
Firebase-Tools

\subsubsection{Firebase Authentication}
\author{Sandin Habibovic}
Email und Passwort authentication

\subsubsection{Cloud Firestore}
\author{Sandin Habibovic}
NoSQL-Database, Dokument-basierte Speicherung, Subcollections
\subsubsection{Cloud Storage}
\author{Sandin Habibovic}
Speicherung von Profilbilder und Post-Bilder
\subsubsection{Firebase Cloud Functions}
jeder schreibt über seine CF

\subsection{Algolia Search}
\subsection{Algolia SDK}

\subsubsection{Typesense Search}
\subsection{Typesense SDK}
