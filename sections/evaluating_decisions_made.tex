\section{Zukünftige mögliche Implementierungen}
\setauthor{martin Hausleitner}

Das Ziel des Teams besteht darin, die App jedem Österreicher und jeder Österreicherin zugänglich zu machen. Daher wurden folgende Ideen während des Brainstormings entwickelt:

\subsection{Benutzererfahrung (UX) und Benutzeroberfläche (UI)}
\begin{itemize}
    \item \textbf{UI/UX-Überarbeitung:} Optimierung der Benutzeroberfläche und Benutzererfahrung für eine intuitivere und ansprechendere App-Nutzung.
    \item \textbf{Vereinfachung der Beitragserstellung:} Einfacheres Erstellen von Beiträgen durch verbesserte Benutzerführung.
    \item \textbf{Automatische Kategorisierung von Beiträgen:} Einsatz von KI, um Beiträge automatisch passenden Kategorien zuzuordnen.
    \item \textbf{Anpassbare Benachrichtigungseinstellungen:} Benutzer können individuelle Einstellungen für Benachrichtigungen festlegen.
\end{itemize}

\subsection{Registrierung und Profil}
\begin{itemize}
    \item \textbf{Vereinfachung des Registrierungsprozesses:} Adressen-Autocomplete-Funktion, um den Registrierungsprozess schneller und einfacher zu gestalten.
    \item \textbf{Identitätsverifizierung:} Optionale Verifizierung der Benutzeridentität anhand eines Ausweises, um Vertrauen innerhalb der Community aufzubauen.
\end{itemize}

\subsection{Kommunikation und Interaktion}
\begin{itemize}
    \item \textbf{Neuentwicklung des Chats:} Einführung von Gruppenchats, Chat-Reaktionen, Sprachnachrichten und erweiterten Anhangsoptionen.
    \item \textbf{Punktesystem:} Belohnung von aktiven Nachbarn durch ein Karma-ähnliches Punktesystem.
    \item \textbf{Integration von Veranstaltungskalendern:} Hinzufügen von lokalen Events und Veranstaltungskalendern zur App, um die Nachbarschaftsinteraktion zu fördern.

\end{itemize}

\subsection{Hilfe und Support}
\begin{itemize}
    \item \textbf{Erweiterte Hilfe und Tutorials:} Bereitstellung von umfassenden Hilfeinhalten und Tutorials, um Benutzern den Einstieg zu erleichtern.
\end{itemize}

\subsection{Technische Verbesserungen}
\begin{itemize}
    \item \textbf{Umfangreiches Caching:} Implementierung von Caching-Strategien, um die App-Performance zu verbessern und die Ladezeiten zu reduzieren.
    \item \textbf{Chat-Verschlüsselung:} Einführung einer Ende-zu-Ende-Verschlüsselung für Chats, um die Privatsphäre und Sicherheit der Benutzer zu gewährleisten.
    \item \textbf{Deep Links:} Ermöglichen von Deep Links für jede Seite, um das Teilen von Inhalten und die Benutzerfreundlichkeit zu erhöhen.
\end{itemize}

\subsection{Sicherheit und Datenschutz}
\begin{itemize}
    \item \textbf{Erweiterte Sicherheitsmechanismen:} Implementierung zusätzlicher Sicherheitsfunktionen, um die App vor Bedrohungen zu schützen.
    \item \textbf{Erweiterte Benachrichtigungsoptionen:} Verbesserung der Benachrichtigungsoptionen, um Benutzer über wichtige Sicherheits- und Datenschutzupdates zu informieren.
\end{itemize}



\subsection{Meilensteine}
\setauthor{Sandin Habibovic}

Zu Beginn der Diplomarbeit wurden folgenden Meilensteine festgelegt:
\\\\
\begin{tabular}{|c|p{10cm}|}
    \hline
    13.03.2022 & Alle Funktionen sind vollständig implementiert                                                                                                                                                                           \\
    10.07.2022 & Ein Interface-Prototyp ist designt                                                                                                                                                                                       \\
    17.07.2022 & Die Systemarchitektur ist definiert und die Machbarkeit geprüft                                                                                                                                                          \\
    07.08.2022 & Ein Minimum Viable Product, welches die Funktionen Login, Feed, Post erstellen, Filter und Chat-Funktion beinhaltet, ist implementiert                                                                                   \\
    07.01.2023 & Ein Minimum Loveable Product, welches die Funktionen Registrierung, Suche, Kontoeinstellungen, Benachrichtigungs-Tab, Übersetzung, Kommentarfunktion, mehrere Kategorien und Report-System beinhaltet, ist implementiert \\
    26.02.2023 & Alle geplanten Funktionen sind vollständig implementiert                                                                                                                                                                 \\
    27.03.2023 & Unvollkommenheiten sind durch Kunden-Feedback und Bugfixing behoben                                                                                                                                                      \\
    01.04.2023 & Diplomarbeit ist vollständig abgeschlossen und abgegeben                                                                                                                                                                 \\
    \hline
\end{tabular}

Unter den Meilensteinen wurden die wichtigsten Etappen des Projekts definiert. Allerdings konnten die festgelegten Termine, wegen Überschätzung des Arbeitsaufwandes, nicht eingehalten werden.
\\
Trotz der Verzögerungen konnten jedoch das Minimum Viable Product und das Minimum Loveable Product erfolgreich fertiggestellt werden.
\\
In den Wochen vor der Abgabe wurden noch Bugs gesucht und behoben, um die Qualität der App weiter zu verbessern.
\\
Da die App erst sehr spät auf dem Playstore veröffentlicht wurde, ist es dem Team zum Zeitpunkt der Abgabe nicht gelungen, Kundenfeedback einzuholen.


\section{Erfahrungen}
\setauthor{Sandin Habibovic}

Während der Erstellung der Diplomarbeit hat das Team wertvolle Erfahrungen gesammelt und umfangreiches Wissen in verschiedenen Technologien erworben. Die Erfahrungen lassen sich in verschiedene Bereiche aufteilen:

\subsection{Frontend-Entwicklung mit Flutter}
\setauthor{Sandin Habibovic}

Die Diplomarbeit bot dem Team die Möglichkeit, sich intensiv mit der Frontend-Entwicklung auseinanderzusetzen und dabei die Vorteile von Flutter als Framework zu erkennen. Da die meisten Teammitglieder bis dato keine oder nur wenig Erfahrung in diesem Bereich hatten, stellte dies eine wertvolle Lernerfahrung dar.
\\
Die Teammitglieder eigneten sich Kenntnisse über den grundsätzlichen Aufbau von Flutter-Anwendungen an, wie zum Beispiel die Verwendung von Widgets, State Management und das Implementieren von Animationen. Die Herausforderung bestand darin, die Funktionalität der Anwendung in einer strukturierten und effizienten Weise zu entwickeln, während gleichzeitig ein ansprechendes Design und Benutzeroberfläche beibehalten wurde.

\subsection{Backend-as-a-Service mit Firestore}
\setauthor{Sandin Habibovic}

Das Team konnte wertvolle Erfahrungen im Umgang mit einer BaaS wie Firestore sammeln.
\\
Die Teammitglieder lernten, wie sie Firestore-Dokumente und -Sammlungen erstellen, abfragen und manipulieren können, um die erforderlichen Daten für die Anwendung bereitzustellen. Außerdem erwarb das Team Kenntnisse darüber, wie Security Rules effektiv formuliert und angewendet werden können.
\\
Darüber hinaus ermöglichte das Arbeiten mit Cloud Storage den Teammitgliedern, Erfahrungen im Umgang mit Dateiuploads und -downloads zu sammeln, während Cloud Functions dazu beitrugen, serverseitige Funktionen für die Anwendung besser zu verstehen.

\subsection{Teamarbeit außerhalb der Schule}
\setauthor{Sandin Habibovic}

Die Diplomarbeit stellte für das Team auch eine Gelegenheit dar, Teamarbeit außerhalb des schulischen Rahmens zu erfahren. Die Teammitglieder lernten, wie sie effektiv zusammenarbeiten, Kommunikationskanäle einrichten und Ressourcen verwalten können, um ein erfolgreiches Projekt abzuschließen. Die Zusammenarbeit erforderte die Entwicklung von Soft Skills wie Zeitmanagement, Planung und Organisation, sowie die Fähigkeit, Feedback anzunehmen und konstruktive Kritik zu üben.
\\
Die Teammitglieder nutzten verschiedene Projektmanagement-Tools und Kommunikationsplattformen, um den Arbeitsfortschritt zu verfolgen, Aufgaben zuzuweisen und Probleme gemeinsam zu lösen. Dies führte zu einer verbesserten Effizienz und half dabei, den Fokus auf die wichtigsten Aspekte des Projekts zu legen.
\\
Zudem lernten die Teammitglieder, wie sie ihre individuellen Stärken und Fähigkeiten am besten einsetzen können, um das Projekt voranzubringen. Durch die Verteilung der Verantwortlichkeiten und die Arbeitsteilung konnte das Team sicherstellen, dass jeder Beitrag zum Gesamterfolg des Projekts beitrug.

\subsection{Teilnahme an Wettbewerben}
\setauthor{Sandin Habibovic}

Die Erfahrungen, die das Team während der Diplomarbeit gesammelt hat, wurden auch durch die Teilnahme an Wettbewerben erweitert und gefestigt. Die Teilnahme an solchen Veranstaltungen ermöglichte den Teammitgliedern, ihr Wissen und ihre Fähigkeiten in einem wettbewerbsorientierten Umfeld zu testen und ihre Arbeit gegenüber anderen Projekten zu validieren.
\\
Die Teilnahme an Wettbewerben bot dem Team die Möglichkeit, von der Expertise der Jury und anderen Teilnehmern zu profitieren und wertvolles Feedback für die Weiterentwicklung und Verbesserung des Projekts zu erhalten.
\\
Des Weiteren trug die Teilnahme an Wettbewerben zur Motivation und zum Zusammenhalt des Teams bei. Die Anerkennung durch Fachleute und die Möglichkeit, Preise und Auszeichnungen zu gewinnen, spornten die Teammitglieder an, ihr Bestes zu geben und kontinuierlich an der Optimierung des Projekts zu arbeiten.
